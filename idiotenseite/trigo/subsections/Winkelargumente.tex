\subsection{Funktionswerte für Winkelargumente}
	\begin{multicols}{4}	
	\begin{tabular}[c]{|p{0.5cm}|p{0.4cm}||p{0.5cm}|p{0.5cm}|p{0.5cm}|}
	    	\hline
			deg & rad & sin & cos & tan\\
			\hline
			0\symbol{23} & 0 & 0 & 1 & 0\\
			\hline
			30\symbol{23} & $\frac{\pi}{6}$ & $\frac{1}{2}$ & $\frac{\sqrt{3}}{2}$ &
			$\frac{\sqrt{3}}{3}$\\
			\hline
			45\symbol{23} & $\frac{\pi}{4}$ & $\frac{\sqrt{2}}{2}$ & $\frac{\sqrt{2}}{2}$
			& 1\\
			\hline
			60\symbol{23} & $\frac{\pi}{3}$ & $\frac{\sqrt{3}}{2}$ & $\frac{1}{2}$ &
			$\sqrt{3}$\\
			\hline			
	\end{tabular} \\
	
	\begin{tabular}[c]{|p{0.7cm}|p{0.7cm}||p{0.7cm}|p{0.7cm}|}
	    	\hline
			deg & rad & sin & cos\\
			\hline
			90\symbol{23} & $\frac{\pi}{2}$ & 1 & 0\\
			\hline	
			120\symbol{23} & $\frac{2\pi}{3}$ & $\frac{\sqrt{3}}{2}$ & $-\frac{1}{2}$ \\
			\hline
			135\symbol{23} & $\frac{3\pi}{4}$ & $\frac{\sqrt{2}}{2}$ & $-\frac{\sqrt{2}}{2}$\\
			\hline
			150\symbol{23} & $\frac{5\pi}{6}$ & $\frac{1}{2}$ & $-\frac{\sqrt{3}}{2}$\\
			\hline
	\end{tabular} \\
	
	\begin{tabular}[c]{|p{0.7cm}|p{0.7cm}||p{0.7cm}|p{0.7cm}|}
	  	\hline
		deg & rad & sin & cos\\
		\hline
		180\symbol{23} & $\pi$ & 0 & -1\\
		\hline	
		210\symbol{23} & $\frac{7\pi}{6}$ & $-\frac{1}{2}$ & $-\frac{\sqrt{3}}{2}$\\
		\hline
		225\symbol{23} & $\frac{5\pi}{4}$ & $-\frac{\sqrt{2}}{2}$ & $-\frac{\sqrt{2}}{2}$\\
		\hline
		240\symbol{23} & $\frac{4\pi}{3}$ & $-\frac{\sqrt{3}}{2}$ & $-\frac{1}{2}$\\
		\hline
	\end{tabular} \\
	
	\begin{tabular}[c]{|p{0.7cm}|p{0.7cm}||p{0.7cm}|p{0.7cm}|}
    	\hline
		deg & rad & sin & cos\\
		\hline
		270\symbol{23} & $\frac{3\pi}{2}$ & -1 & 0\\
		\hline	
		300\symbol{23} & $\frac{5\pi}{3}$ & $-\frac{\sqrt{3}}{2}$ & $\frac{1}{2}$\\
		\hline
		315\symbol{23} & $\frac{7\pi}{4}$ & $-\frac{\sqrt{2}}{2}$ & $\frac{\sqrt{2}}{2}$\\
		\hline
		330\symbol{23} & $\frac{11\pi}{6}$ & $-\frac{1}{2}$ & $\frac{\sqrt{3}}{2}$\\
		\hline
	\end{tabular}					
\end{multicols}